\section{結論}

\begin{frame}[noframenumbering]{目次}
    \tableofcontents[currentsection]
\end{frame}

\begin{frame}{結論}
    \begin{itemize}
        \item 本研究では,共焦点顕微鏡を用いて得られた\red{複数の焦点面における二次元画像の時系列データ}を入力として,複数のプロトプラスト化した細胞をリアルタイム追跡する手法を開発した.
        \vs
        \item \textbf{Depth-SORT}は\red{スライスを跨いだReID}を可能にし,\textbf{Slice Kalman Filter}は複数の焦点面におけるバウンディングボックス集合から\red{三次元的な位置の推定}を可能にした.また検証用データを用いた評価により,これらが時間的・空間的に疎なデータから高精度に三次元位置を推定・予測できることを確認した.
        \vs
        \item さらに実際にシロイヌナズナの根を対象として,この複数物体追跡手法を組み込んだ\red{位置情報付き一細胞分取システムによる自動分取に成功}した.このシステムが一細胞レベルの空間分解能を持ったscRNA-seq,ひいては植物リプログラミングの解明に貢献できることを信じている.
    \end{itemize}
\end{frame}
